%Carattere dimensione 12
\documentclass[12pt]{report}

%Margini e interlinea
\usepackage[top=1in, bottom=1in, left=1.2in, right=1in]{geometry}
\pagestyle{plain}
\linespread{1.5}

%Librerie utili
\usepackage[italian]{babel}
\usepackage[utf8]{inputenc}
\usepackage{graphicx}
\usepackage{floatflt}
\usepackage{blindtext}
\usepackage{enumitem}
\usepackage{amsthm}
\usepackage{subfig}
\usepackage{listings}
\usepackage{listingsutf8}
\usepackage{amsmath}
\usepackage{framed}
\usepackage{minibox}
\usepackage{float}
\usepackage{wrapfig}
\usepackage{longtable}
\usepackage[strict]{changepage}
\usepackage{pgfplots}
\usepackage{tikz}
\usetikzlibrary{matrix}
\pgfplotsset{width=11cm,compat=1.9}
\usepgfplotslibrary{external}
\tikzexternalize


% Code Listings
\definecolor{vgreen}{RGB}{104,180,104}
\definecolor{vblue}{RGB}{49,49,255}
\definecolor{vorange}{RGB}{255,143,102}
\lstdefinestyle{bash} {
	language=bash,
	basicstyle=\ttfamily,
	keywordstyle=\color{vblue},
	identifierstyle=\color{black},
	commentstyle=\color{vgreen},
	tabsize=4,
%	moredelim=*[s][\colorIndex]{[}{]},
	literate=*{:}{:}1
}

\begin{document}

\begin{titlepage}
\begin{figure}[t]
	\centering\includegraphics[width=0.9\textwidth]{scritta}
    \centering\includegraphics[width=0.4\textwidth]{logo}
\end{figure}

\begin{center}
	\textbf{ Dipartimento di Informatica\\ Corso di Laurea in Informatica\\}
	\vspace{15mm}
    {\LARGE{\bf Relazione del progetto midterm in Java per il corso di Programmazione 2}}\\
	\vspace{3mm}
	{\LARGE{\bf A cura di Alessandro Cheli}}\\
\end{center}

\vspace{18mm}

\begin{minipage}[t]{0.47\textwidth}
	{\large{\bf Insegnanti:\\ Prof. Gianluigi Ferrari\\ Prof.ssa Francesca Levi}}
\end{minipage}\hfill\begin{minipage}[t]{0.47\textwidth}\raggedleft
	{\large{\bf Studente: \\Alessandro Cheli \\ 583350 Corso A\\ }}
\end{minipage}

\vspace{18mm}

\centering{\large{\bf Sessione autunnale\\ Anno Accademico 20190/2020 }}

\end{titlepage}

\chapter{Relazione del Progetto}

\section{Istruzioni}

Il progetto è stato realizzato utilizzando \textbf{Apache Maven} come strumento di build e gestione delle dipendenze.
Alcune clausole sono specificate con la sintassi \textbf{Javadoc}
(i parametri sono documentati con la clausola \textit{@params} invece che con la clausola \textit{@requires}).
Dopo aver installato JDK di versione uguale o superiore alla 8 e la dipendenza Apache Maven
si può compilare ed eseguire il progetto eseguendo.
\begin{lstlisting}[style=bash]
mvn compile
mvn package
java -jar target/application-1.0-SNAPSHOT.jar
\end{lstlisting}

In alternativa si può compilare il progetto utilizzando javac ma non verranno spiegati nella relazione i dettagli.
Apache Maven scaricherà e compilerà automaticamente le dipendenze necessarie (per generare Javadoc e creare un archivio \verb|.jar| corretto).
Non vengono utilizzate librerie esterne a \verb|java.util|.

Per visualizzare la documentazione \textbf{Javadoc} si può eseguire, dopo la compilazione
\begin{lstlisting}[style=bash]
# entro nella directory dove risiedono i file compilati
cd application/target
# creo una cartella dove estrarre i file html
mkdir javadoc
# estraggo il contenuto dell'archivio jar dove
# sono contenuti i Javadoc
unzip application-1.0-SNAPSHOT-javadoc.jar -d javadoc/
# eseguire un server http all'interno della
# directory contenente i file html
# ad esempio darkhttpd
darkhttpd javadoc/
darkhttpd/1.12, copyright (c) 2003-2016 Emil Mikulic.
listening on: http://0.0.0.0:8080/
# aprire un browser all'indirizzo localhost:8080
\end{lstlisting}

\section{Dettagli di Implementazione}
La classe \verb|Board<E extends DataElement>| implementa l'interfaccia \\
\verb|DataBoard<E extends DataElement>|
mantenendo i contenuti (post) all'interno di una tabella hash \verb|private HashMap<String, TreeSet<E>> contents;|.
Dove la chiave è la categoria e il contenuto è un insieme di post ordinato per numero di like e lessicograficamente.
(Si veda \verb|DataElement.compareTo| per l'ordinamento dei post). Gli amici vengono
contenuti in una tabella hash \verb|private HashMap<String, HashSet<String>> friends;|
Dove la chiave è la categoria e i valori sono gli insiemi di amici che possono visualizzare tale categoria.

La seconda implementazione \verb|Board2<E extends DataElement>| utilizza una classe
\verb|Category<E extends DataElement>| che contiene, oltre al nome
della categoria, un \verb|TreeSet<E>| per mantenere i post ordinati e
un \verb|HashSet<String>| per mantenere i nomi degli amici che possono visualizzare la categoria.
Nella classe \verb|Board2| si usa una \verb|ArrayList<Category>| per mantenere la lista delle categorie e si effettua il
controllo dell'unicità a mano sul campo \verb|c.category| $\forall$ \verb|Category c| tale che \verb|c| è contenunta all'interno
della board.

Entrambe le implementazioni effettuano rigorosi controlli sull'unicità, clonazione e correttezza dei valori dove tali attributi sono richiesti.
In un'applicazione reale si utilizzerebbe, oltre che ad un database \textit{SQL} per la permanenza dei dati serializzati, anche dei
metodi di crittografia come \textit{BCRYPT+Salt} per la gestione delle password e dei contenuti, ed un server che espone endpoint di una \textit{API HTTP}.
Per semplicità, il progetto non richiede nessuna di tali funzionalità e l'implementazione e la batteria di test non gestiscono crittografia o permanenza dei dati.

\section{Dettagli sulla batteria di test}
La batteria di test è realizzata per semplicità \textbf{a mano} nel metodo main
e non con \textbf{Junit}. Viene controllata la correttezza delle eccezioni lanciate all'interno
di \textit{worst case} appositi dove si intende ottenere tali eccezioni, e vengono eseguite operazioni comuni
su tutti i metodi di entrambe le classi \verb|Board| e \verb|Board2|. Nel caso le operazioni
di \textit{normal usage} lancino un eccezione non aspettata viene terminata l'esecuzione del programma di test con un messaggio
di \textit{FATAL ERROR}. Se un eccezione lanciata in un \textit{worst case}
non rispetta il tipo di eccezione che si aspettava viene riportato un messaggio \textit{FAIL}, altrimenti un messaggio \textit{PASS}. Per controllare che l'esecuzione del main di test non contenga messaggi \textit{FAIL} o \textit{FATAL} si può eseguire

\begin{lstlisting}[style=bash]
java -jar application/target/application-1.0-SNAPSHOT.jar 2>&1 \
	| grep -F -e FAIL -e FATAL
\end{lstlisting}

\appendix

%\bibliography{references}
%\bibliographystyle{plain}

\end{document}